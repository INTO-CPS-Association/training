\chapter{Concepts and Terminology}
%\fbox{UNEW: RP}
\label{sec:concepts}

This section introduces the basic concepts used in the INTO-CPS project. CPSs bring together domain experts from  diverse backgrounds, from software engineering to control engineering. Each discipline has developed their own terminologies, principles and philosophy for years --- in places they use similar terms for quite different meanings and different terms that have the same meaning. In addition, the \into\ project aims to produce a tool chain for CPS engineering resulting in the need for common tool-based terminology. \into\ requires experts from diverse fields to work collaboratively, so this section gives some core concepts of \into\  that will be used throughout the project. We divide the concepts into several broad areas in the remainder of this section.

\section{Systems}
\label{sec:concepts:systems}
A \term{System} is defined as being ``a combination of interacting elements organized to achieve one or more stated purposes''~\cite{INCOSEseh15}. Any given system will have an~\term{environment}, considered to be everything outside of the system. The behaviour exhibited by the environment is beyond the direct control of the developer~\cite{Broenink&12b}. We also define a \term{system boundary} as being the common frontier between the system and its environment. The definition of the system boundary is application-specific~\cite{Broenink&12b}.

\term{Cyber-Physical Systems (CPSs)} refer to ``ICT systems (sensing, actuating, computing, communication, etc.) embedded in physical objects, interconnected (including through the Internet) and providing citizens and businesses with a wide range of innovative applications and services''~\cite{Thompson13, Deka&15}.
%\fbox{``Complex CPS'' used in Part B of DoW -- is the degree of complexity important?}

A \term{System of Systems (SoS)} is a ``collection of constituent systems that pool their resources and capabilities together to create a new, more complex system which offers more functionality and performance than simply the sum of the constituent systems''~\cite{Holt&14}. CPSs may exhibit the characteristics of SoSs.

\section{Models}
\label{sec:concepts:models}

In the \into\ project, we concentrate on ``model-based design'' of CPSs. A \term{model} is a  potentially partial and abstract description of a system, limited to those components and properties of the system that are pertinent to the current goal~\cite{Holt&14}. A model should be ``just complex enough to describe or study the phenomena that are relevant for our problem context''~\cite{Amerongen10}. Models should be abstract ``in the sense that aspects of the product not relevant to the analysis in hand are not included''~\cite{Fitzgerald&98b}. A model ``may contain representations of the system, environment and stimuli''~\cite{Fitzgerald&14c}\footnote{Further discussion is required in the final year of INTO-CPS regarding the definition of aspects of models in particular; environment models, test models in RT-Tester and their correspondence in the INTO-CPS SysML profile.}.

In a CPS model, we model systems with cyber, physical and network elements. These components are  often drawn from different domains, and are modelled in a variety of languages, with different notations, concepts, levels of abstraction, and semantics, which are not necessarily easily mapped one to another. This heterogeneity presents a significant challenge for simulation in CPSs~\cite{Holt&14}. In \into\ we use \term{continuous time (CT)} and \term{discrete event (DE)} models to represent physical and cyber elements as appropriate. A CT model has state that can be changed and observed \emph{continuously}~\cite{Amerongen10} and is described using either explicit continuous functions of time  either implicitly as a solution of differential equations. A DE model has state that can be changed and observed only at fixed, \emph{discrete}, time intervals~\cite{Amerongen10}.  The approach used in the DESTECS project was to use \emph{co-models} -- ``a model comprising a DE model, a CT model and a contract''~\cite{Broenink&12b}. In \into\ we propose the use of \term{multi-models} -- ``comprising multiple \term{constituent} DE and CT models''. Related to this is a \term{Hybrid Model}, which contains both DE and CT elements.

A \term{requirement} may impose restrictions, define system capabilities or identify qualities of a system and should indicate some value or use for the different stockholders of a CPS. \term{Requirements Engineering (RE)} is the process of the specification and documentation of requirements placed upon a CPS. Requirements may be considered in relation to different \term{contexts} -- that is the point of view of some system component or domain, or interested stakeholder.

We cover the main features of the notations used in \into\ in Section~\ref{sec:concepts:language}. Here we consider some general terms used in models. A \term{design parameter} is a property of a model that can be used to affect the model's behaviour, but remains constant during a given simulation~\cite{Broenink&12b}. A \term{variable} is feature of a model that may change during a given simulation~\cite{Broenink&12b}. \term{Non-functional properties (NFPs)} pertain to characteristics other than functional correctness. For example, reliability, availability, safety and performance of specific functions or services are NFPs that are quantifiable. Other NFPs may be more difficult to measure~\cite{Payne&10}.

The activity of creating models may be referred to as \term{modelling} ~\cite{Fitzgerald&14c} and related terms include \term{co-modelling} and \term{multi-modelling}. A \term{workflow} is a sequence of \term{activities} performed to aid in modelling. A workflow has a defined purpose, and may cover a subset of the CPS engineering development lifecycle.

The term \term{architecture} has many different definitions, and range in scope depending upon the scale of the product being `architected'. In the \into\ project, we use the simple definition from~\cite{COMPASSD22.6}:  ``an architecture defines the major elements of a system, identifies the relationships and interactions between the elements and takes into account process. Those elements are referred to as \term{components}. An architecture involves both a definition of structure and behaviour. Importantly, architectures are not static but must evolve over time to reflect the change in a system as it evolves to meet changes to its requirements''. In a CPS architecture, components may be either \term{cyber components} or \term{physical components} corresponding to some functional logic or an entity of the physical world respectively.

In \into\ we consider both a \term{holistic architecture} and a \term{design architecture}. An example of their use is given in Chapter~\ref{sec:sysml}. The aim of a holistic architecture is to identify the main units of functionality of the system reflecting the \emph{terminology and structure of the domain of application}. It describes a conceptual model that highlights the main units of the system architecture and the way these units are connected with each other, taking a holistic view of the overall system. The design architectural model of the system is effectively a multi-model. The INTO-CPS SysML profile~\cite{INTOCPSD2.1a} is designed to enable the specification of CPS design architectures, which emphasises a decomposition of a system into \term{subsystems}, where each subsystem is an assembly of cyber and physical components and possibly other subsystems, and modelled separately in isolation using a special notation and tool designed for the domain of the subsystem. \term{Evolution} refers to the ability of a system to benefit from a varying number of alternative system components and relations, as well as its ability to gain from the adjustments of the individual components' capabilities over time (Adjusted from SoS~\cite{Nielsen&13}).

Considering the interactions between components in a system architecture, an \term{interface} ``defines the boundary across which two entities meet and communicate with each other''~\cite{Holt&14}. Interfaces may describe both digital and physical interactions: digital interfaces  contain descriptions of operations and attributes that are \emph{provided} and \emph{required} by components. Physical interfaces describe the flow of physical matter (for example fluid and electrical power) between components.

There are many methods of describing an architecture. In the \into\ project, an \term{architecture diagram} refers to the symbolic representation of architectural information contained in a model. An \term{architectural framework} is a ``defined set of viewpoints and an ontology'' and ``is used to structure an architecture from the point of view of a specific industry, stakeholder role set, or organisation.~\cite{Holt&14}. In the application of an architecture framework, an \term{architectural view} is a ``work product (for example an architecture diagram) expressing the architecture of a system from the perspective of specific system concerns''~\cite{COMPASSD22.6}.

The \into\ SysML profile comprises diagrams for architectural modelling and \term{design space exploration} specification. There are two architectural diagrams. The \term{Architecture Structure Diagram (ASD)} specialises SysML block definition diagrams to support the specification of a system architecture described in terms of a system's components. \term{Connections Diagrams (CDs)} specialise SysML internal block diagrams to convey the internal configuration of the system's components and the way they are connected. The system architecture defined in the profile should inform a co-simulation multi-model and therefore all components interact through connections between flow ports. The profile permits the specification of \term{cyber} and \term{physical} components and also components representing the \term{environment} and \term{visualisation} elements. The \into\ SysML profile includes three design space exploration diagrams: a \term{parameters diagram}; an \term{objective diagram}; and a \term{ranking diagram}. See Section~\ref{sec:concepts:analysis} for concepts relating to design space exploration.


\section{Tools}
\label{sec:concepts:tools}

The \term{\into\ tool chain} is a collection of software tools, based centrally around FMI-compatible co-simulation, that  supports the collaborative development of CPSs. The \term{\into\ Application} is a front-end to the INTO-CPS tool chain. The application allows the specification of the co-simulation configuration, and the co-simulation execution itself. The application also provides access to features of the tool chain without an existing user interface (such as design space exploration and model checking). Central to the \into\ tool chain is the use of the Functional Mockup Interface (FMI) standard.

The \term{Functional Mockup Interface (FMI)} is a tool-independent standard to support both model exchange and co-simulation of dynamic models using a combination of XML-files and compiled C-code~\cite{FMIStandard2.0}. Part of the FMI standard for model exchange is specification of a \term{model description} file. This is an XML file that supplies a description of all properties of a model (for example input/output variables). A \term{Functional Mockup Unit (FMU)} is a tool component that implements FMI. Data exchange between FMUs and the synchronisation of all simulation solvers~\cite{FMIStandard2.0} is controlled by a \term{Master Algorithm}.

\term{Co-simulation}  is the simultaneous, collaborative, execution of models and allowing information to be shared between them. The models may be CT-only, DE-only or a combination of both. The \term{Co-simulation Orchestration Engine (COE)} combines existing co-simulation solutions (FMUs) and scales them to the CPS level, allowing CPS multi-models to be evaluated through co-simulation. This means that the COE implements a \term{Master Algorithm}. The COE will also allow real software and physical elements to participate in co-simulation alongside models, enabling both Hardware-in-the-Loop (HiL) and Software-in-the-Loop (SiL) simulation.

In the \into\ Application, a \term{project} comprises: a number of FMUs, optional source models (from which FMUs are exported); a collection of \term{multi-models}; and an optional SysML architectural model. A multi-model includes a list of FMUs, defined instances of those FMUs, specified connections between the inputs/outputs of the FMU instances, and defined values for design parameters of the FMU instances. For each multi-model a \term{co-simulation configuration} defines the step size configuration, start and end time for the co-simulation of that multi-model. Several configurations can be defined for each multi-model.

\term{Code generation} is the transformation of a model into generated code suitable for compilation into one or more target languages (e.g. C or Java).

The \into\ project considers two tool-supported methods for recording the rationale of design decisions in CPSs.  \term{Traceability} is the association of one model element (e.g. requirements, design artefacts, activities, software code or hardware) to another. \term{Requirements traceability} ``refers to the ability to describe and follow the life of a requirement, in both a forwards and backwards direction''~\cite{Gotel&94}. \term{Provenance} ``is information about entities, activities, and people involved in producing a piece of data or thing, which can be used to form assessments about its quality, reliability or trustworthiness'' ~\cite{Moreau&13}. In \into\ traceability between model elements defined in the various modelling tools is achieved through the use of \term{OSLC messages}, handled by a traceability \term{daemon tool}. This supports the \term{impact analysis} and general \term{traceability queries}.

Two broad groups of users are considered in the \into\ project. A \term{Tool Chain User} is an individual who uses the \into\ Tool Chain and its various analysis features. A \term{Foundations Developer} is someone who uses the developed foundations and associated tool support (see Section~\ref{sec:concepts:formal}) to reason about the development of tools.

\section{Analysis}
\label{sec:concepts:analysis}

\term{Design-Space Exploration (DSE)} is ``an activity undertaken by one or more engineers in which they build and evaluate [multi]-models in order to reach a design from a set of requirements''~\cite{Broenink&12b}. ``The \term{design space} is the set of possible solutions for a given design problem''~\cite{Broenink&12b}. Where two or more models represent different possible solutions to the same problem, these are considered to be \term{design alternatives}. In \into\, design alternatives are defined using either a range of parameter values or different multi-models. Each choice involves making a selection from alternatives on the basis of an \term{objective} -- criteria or constraints that are important to the developer, such as cost or performance. The alternative selected at each point constrains the range of design alternatives that may be viable next steps forward from the current position. Given a collection of alternatives with corresponding objective results, a \term{ranking} may be applied to determine the `best' design alternative.

\term{Test Automation (TA)} is defined as the machine assisted automation of system tests. In \into, we concentrate on various forms of \term{model-based testing} -- centering on testing system models, against the requirements on the system. The \term{System Under Test (SUT)} is ``the system currently being tested for correct behaviour. An alias for system of interest, from the point of view of the tester''~\cite{Holt&14}. The SUT is tested against a collection of \term{test cases} --  a finite structure of input and expected output~\cite{Utting&06}, alongside a \term{test model}, which specifies the expected behaviour of a system under test~\cite{Coleman&13b}. TA uses a \term{test suite} -- a collection of \term{test procedures}. These test procedures are detailed instructions for the set-up and execution of a given set of test cases, and instructions for the evaluation of results of executing the test cases~\cite{DO178B}.

\into\ considers three main types of test automation: \term{Hardware-in-the-Loop (HiL)}, \term{Software-in-the-Loop (SiL)} and \term{Model-in-the-Loop (MiL)}. In \term{HiL} there is (target) hardware involved, thus the FMU is mainly a wrapper that interacts (timed) with this hardware; it is perceivable that realisation heavily depends on hardware interfaces and timing properties.
%testing with DE models running on target hardware components;
In \term{Software-in-the-Loop (SiL)} testing the object of the test execution is an FMU that contains a software implementation of (parts of) the system. It can be compiled and run on the same machine that the COE runs on and has no (defined) interaction other than the FMU-interface.
%����It does not matter (much) where this implementation comes from.
%testing with software running on CT model simulator;
Finally, in \term{Model-in-the-Loop (MiL)} the test object of the test execution is a (design) model, represented by one or more FMUs. This is similar to the SiL (if e.g., the SUT is generated from the design model), but MiL can also imply that running the SUT-FMU has a representation on model level; e.g., a playback functionality in the modelling tool could some day be used to visualise a test run.
%testing with co-simulated CT/DE models. \fbox{Check definitions}

\term{Model Checking (MC)} exhaustively checks whether the model of the system meets its specification~\cite{Clarke&99}, which is typically expressed in some temporal logic such as \term{Linear Time Logic (LTL)}~\cite{Pnueli77} or \term{Computation Tree Logic (CTL)}~\cite{Clarke&81}. As opposed to testing, model checking examines the entire state space of the system and is thus able to provide a correctness proof for the model with respect to its specification. In INTO-CPS, we can concentrate on \term{Bounded Model Checking (BMC)}~\cite{Clarke&01,Clarke&04,Clarke&05}, which is based on encodings of the system in propositional logic, for a timed variant of LTL. The key idea of this approach is to represent the semantics of the model as a Boolean formula and then apply a \term{Satisfiability Modulo Theory (SMT)}~\cite{Kroening&08} solver in order to check whether the model satisfies its specification. A powerful feature of model checking is that, if the specification is violated, it provides a counterexample trace that shows exactly how an undesired state of the system can be reached~\cite{Clarke&03}.

\section{Existing Tools and Languages}
\label{sec:concepts:language}

The \into\ tool chain uses several existing modelling tools. \term{Overture}\footnote{\url{http://overturetool.org/}} supports modelling and analysis in the design of discrete, typically, computer-based systems using the \term{VDM-RT} notation. VDM-RT is based upon the \term{object-oriented} paradigm where a model is comprised of one or more \term{objects}. An object is an instance of a \term{class} where a class gives a definition of zero or more \term{instance variables} and \term{operations} an object will contain. Instance variables define the identifiers and types of the data stored within an object, while operations define the behaviours of the object.

The \term{20-sim}\footnote{\url{http://www.20sim.com/}} tool can represent continuous time models in a number of ways. The core concept is that of connected \term{blocks}.   \term{Bond graphs} may implement blocks. Bond graphs offer a domain-independent description of a physical system's dynamics, realised as a directed graph. The vertices of these graphs are idealised descriptions of physical phenomena, with their edges (\term{bonds}) describing energy exchange between vertices. Blocks may have input and output \term{ports} that allow data to be passed between them. The energy exchanged in 20-sim is the product of \term{effort} and \term{flow}, which map to different concepts in different domains, for example voltage and current in the electrical domain.

\term{OpenModelica}\footnote{\url{https://www.openmodelica.org/}}  is an open-source \term{Modelica}-based modelling and simulation environment. Modelica is an ``object-oriented language for modelling of large, complex, and heterogeneous physical systems''~\cite{Fritzson&98}. Modelica models are described by \term{schematics}, also called \term{object diagrams}, which consist of connected components. Components are connected by ports and are defined by sub components or a textual description in the Modelica language.

\term{Modelio}\footnote{\url{http://www.modelio.org/}} is an open-source modelling environment supporting industry standards like UML and SysML. \into\ will make use of Modelio for high-level system architecture modelling using the \term{SysML} language and proposed extensions for CPS modelling. The systems modelling language (SysML)~\cite{SysML12}  extends a subset of the UML to support modelling of heterogeneous systems.

\section{Formalisms}
\label{sec:concepts:formal}

The \term{semantics} of a language describes the meaning of a (grammatically correct) program~\cite{Nielson&92} (or model). There are different methods of defining a language semantics: \term{structural operational semantics}; \term{denotational semantics}; and \term{axiomatic semantics}.

A structural operational semantics (SOS) describes how the individual steps of a program are executed on an abstract machine~\cite{Plotkin81}. An SOS definition is akin to an interpreter in that it provides the meaning of the language in terms of relations between beginning and end states. The relations are defined on a per-construct basis. Accompanying the relations are a collection of semantic rules which describe how the end states are achieved. Where an operational semantics defines how a program is executed, a denotational approach defines a language in terms of denotations, in the form of abstract mathematical objects, which represent the semantic function that maps over the inputs and outputs of a program~\cite{Scott&71}.

The Unifying Theories of Programming (UTP)~ \cite{Hoare&98} is a technique to for describing language semantics in a unified framework. A theory of a language is composed of an \term{alphabet}, a \term{signature} and a collection of \term{healthiness conditions}.

The Communicating Sequential Processes \term{CSP} notation~\cite{Hoare85} is a formal process algebra for describing  communication  and interaction.
\term{INTO-CSP} is a version of CSP, which will be used to provide a model for the SysML-FMI profile, FMI, VDM-RT and Modelica semantics. It is a front end for a UTP theory of reactive concurrent continuous systems customised for the needs of INTO-CPS. \term{Hybrid-CSP} is a continuous version of CSP defined originally by He Jifeng~\cite{Jifeng94}. It will be used as a basis to inform the design of INTO-CSP.

Several forms of verification are enabled through the use of formally defined languages.  \term{Refinement} is a verification and formal development technique pioneered by~\cite{Back&98} and~\cite{Morgan90a}. It is based on a behaviour preserving relation that allows the transformation of an abstract specification into more and more concrete models, potentially leading to an implementation. \term{Proof} is the process of showing how the validity of one statement is derived from others by applying justified rules of inference~\cite{Bicarregui&94}.

For the purposes of verification in INTO-CPS, and in particular the work of WP2, we make use of the Isabelle/HOL theorem prover and the FDR3 refinement checker. These are not considered part of the INTO-CPS tool chain, and are used in the INTO-CPS project primarily to support the development of foundation work. 