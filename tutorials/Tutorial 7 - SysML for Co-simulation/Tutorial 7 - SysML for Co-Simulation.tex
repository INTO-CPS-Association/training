\documentclass[11pt,a4paper]{../tutorial}
\usepackage[hidelinks]{hyperref}

\title{Tutorial 7 --- SysML for Co-Simulation}
\date{July 2017}
\author{Etienne Brosse}

\begin{document}

\section*{Overview}

This set of exercises will show you how to:

\begin{enumerate}[noitemsep]
\item Create a simple CPS Design
\item Generate a Co-Simulation
\item Export a FMI Model Description
\item Import a FMI Model Description

\end{enumerate}

\section*{Requirements}

This tutorial requires the following tools from the INTO-CPS tool chain to be installed:

\begin{itemize}[noitemsep]
\item Modelio v3.4.1
\end{itemize}

You may have been provided with tools on a USB drive at your training session. Otherwise the INTO-CPS Application can be downloaded from \url{https://into-cps.github.io/download/} and tools can be downloaded from there through \emph{Window \menusep Show Download Manager} to your \emph{into-cps-projects} install downloads directory. Please ask if you are unsure.

\vspace{4mm}

You may need to update the INTO-CPS extension for Modelio to utilise DSE components. The extension can be downloaded from \mbox{\url{http://forge.modelio.org/projects/intocps-modelio34/files}}. Version 1.2.05 or later is required for DSE modelling.

\newpage

\section{Opening a Project}

\begin{instructions}

\item Launch \emph{Modelio}. On first loading, you may have to close the \emph{Welcome} screen (you can bring it back with `Help \menusep Welcome' if you need)

\begin{center}
\begin{annotation}[width=1\linewidth,trim=0 400 0 0,clip]{figures/modelio_start}
    \ucircle{0.18cm}{0.13}{Close \emph{Welcome} screen}{0.068,0.868}
    \end{annotation}
\end{center}

\item A workspace must be chosen, select `File \menusep Switch Workspace’.

\begin{center}
\begin{annotation}[width=1\linewidth,trim=0 700 0 0,clip]{figures/modelio_workspace}
    \usquare{3.9cm}{0.1cm}{0.25}{Switch Workspace}{0.135,0.44}
%    \helpergrid
    \end{annotation}
\end{center}


\newpage

\item Set the \emph{Workspace} to the location of \emph{tutorial\_7\pathsep{}SysML} and click \emph{Ok}.

\begin{center}
\begin{annotation}[width=0.5\linewidth,trim=0 0 0 0,clip]{figures/modelio_workspace_choice}
    \upoint{0.15}{Path to \emph{tutorial\_8\pathsep{}SysML}}{0.27,0.42}
    \rsquare{1.5cm}{0.15cm}{0.1}{Ok}{0.75,0.06}
%    \helpergrid
    \end{annotation}
\end{center}


\item Left-click on the \emph{LineFollowRobot} model once on the left to see details of the model. Double-click the \emph{LineFollowRobot} model to open the model.

\begin{center}
\begin{annotation}[width=1\linewidth,trim=0 300 0 0,clip]{figures/modelio_select_model}
        \upoint{0.2}{\emph{LineFollowRobot} model}{0.09,0.8}
        \upoint{0.565}{Model Information}{0.565,0.7}
%    \helpergrid
    \end{annotation}
\end{center}

\end{instructions}

 \newpage

\section{Define an Architecture}
\label{sec:obj_def}

This section describes how to the design of the architecture of your system architecture.  CPS Architecture is mainly composed of an unique System Block and a set of CComponent.


\begin{instructions}
\item Right click `$linefollowrobot\_mm$' in model outline. Select `INTO-CPS \menusep Architectural Structure Diagram'.

\begin{center}
\begin{annotation}[width=0.7\linewidth]{figures/modelio_create_architecture_diagram}
    \usquare{3.1cm}{0.06cm}{0.2}{Create an Architectural Structure Diagram}{0.8,0.6}
    %\helpergrid
    \end{annotation}
\end{center}

\newpage

\item From the \emph{Toolbox}, select \emph{System} and click on the empty diagram to create the System block. Double click on the created block and change its name to `$Robot$'.

\begin{center}
\begin{annotation}[width=0.7\linewidth]{figures/modelio_system_block}
    \usquare{2cm}{0.1cm}{0.25}{Create a System Block}{0.13,0.48}
    %\helpergrid
    \end{annotation}
\end{center}

\item \label{start_ccomponent_instructions} From the \emph{Toolbox}, select \emph{CComponent} and add it to the diagram. Double click on the created block and change its name to `$Body$'.

\begin{center}
\begin{annotation}[width=0.7\linewidth]{figures/modelio_ccomponent}
    \usquare{3.2cm}{0.1cm}{0.25}{Create CComponent Body Block}{0.15,0.46}
    %\helpergrid
    \end{annotation}
\end{center}

\item \label{composition_instructions}To associate the '$Body$' block with the Robot, select the \emph{Composition} relation from the \emph{Toolbox}, and click first on the `$Robot$' and then on the `$Body$' block.

\begin{center}
\begin{annotation}[width=0.7\linewidth]{figures/modelio_connect_body}
    \usquare{3.2cm}{0.1cm}{0.25}{Create a Composition Relation}{0.18,0.12}
    %\helpergrid
    \end{annotation}
\end{center}

\item \label{end_ccomponent_instructions} From the \emph{Toolbox}, select \emph{Flow Port} and add it to the `$Body$' CComponent. Select the block that this creates and F2 to change its name to `$robot\_x$'. On the  \emph{INTO-CPS} property view , change \emph{Type} to `$PType::Real$', and \emph{Direction} to `$Out$'.

\begin{center}
\begin{annotation}[width=0.7\linewidth]{figures/modelio_flowport}
    \usquare{1.8cm}{0.1cm}{0.25}{Create a Flow Port}{0.1,0.7}
    %\helpergrid
    \end{annotation}
\end{center}

\item Repeat \ref{end_ccomponent_instructions}  to create the following flow ports to the `$Body$' CComponent:
\begin{itemize}
	\item Name: `$robot\_y$', Type: `$PType::Real$', Direction:  `$Out$';
	\item Name: `$robot\_z$', Type: `$PType::Real$', Direction:  `$Out$';
	\item Name: `$robot\_theta$', Type: `$PType::Real$', Direction:  `$Out$';
	\item Name: `$servo\_right\_input$', Type: `$PType::Real$', Direction:  `$In$';
	\item Name: `$servo\_left\_input$', Type: `$PType::Real$', Direction:  `$In$';
\end{itemize}

\item Repeat from \ref{start_ccomponent_instructions}  to \ref{end_ccomponent_instructions} to create the following ccomponent and their corresponding ports:
\begin{itemize}
	\item Name: `$Sensor1$';
	\begin{itemize}
		\item Name: `$robot\_x$', Type: `$PType::Real$', Direction:  `$In$';
		\item Name: `$robot\_y$', Type: `$PType::Real$', Direction:  `$In$';
		\item Name: `$robot\_z$', Type: `$PType::Real$', Direction:  `$In$';
		\item Name: `$robot\_theta$', Type: `$PType::Real$', Direction:  `$In$';
		\item Name: `$lf\_1\_sensor\_reading$', Type: `$PType::Real$', Direction:  `$Out$';
	\end{itemize}
\end{itemize}

\begin{itemize}
	\item Name: `$Sensor2$';
	\begin{itemize}
	\item Name: `$robot\_x$', Type: `$PType::Real$', Direction:  `$In$';
	\item Name: `$robot\_y$', Type: `$PType::Real$', Direction:  `$In$';
	\item Name: `$robot\_z$', Type: `$PType::Real$', Direction:  `$In$';
	\item Name: `$robot\_theta$', Type: `$PType::Real$', Direction:  `$In$';
	\item Name: `$lf\_1\_sensor\_reading$', Type: `$PType::Real$', Direction:  `$Out$';
\end{itemize}
\end{itemize}


\begin{itemize}
	\item Name: `$Controller$';
	\begin{itemize}
	\item Name: `$ILeftVal$', Type: `$PType::Real$', Direction:  `$In$';
	\item Name: `$IRightVal$', Type: `$PType::Real$', Direction:  `$In$';
	\item Name: `$servo\_right\_out$', Type: `$PType::Real$', Direction:  `$Out$';
	\item Name: `$servo\_left\_out$', Type: `$PType::Real$', Direction:  `$Out$';
\end{itemize}
\end{itemize}

The model should now look like this:

\begin{center}
\begin{annotation}[width=0.7\linewidth]{figures/modelio_architecture_complete}
    %\usquare{3.9cm}{0.1cm}{0.25}{Create Objective Definition Diagram}{0.66,0.58}
    %\helpergrid
    \end{annotation}
\end{center}


\end{instructions}


\newpage

\section{Design a Connection Diagram}

This section will outline the steps necessary to create a Connection Diagram under Modelio.

\begin{instructions}

\item Right click `$linefollowrobot\_mm$' in model outline. Select `INTO-CPS \menusep Connection Diagram'.

\begin{center}
\begin{annotation}[width=0.7\linewidth]{figures/modelio_create_connection_diagram}
    \usquare{3.2cm}{0.05cm}{0.3}{Create Connection Diagram}{0.79,0.56}
    %\helpergrid
    \end{annotation}
\end{center}

\item Select the Block Instance automatically created with the Connection Diagram. On the \emph{INTO-CPS} property view , change \emph{Name} to `$robot2Sensor$', and its \emph{Type} to `$Robot$'.

\item \label{start-instanciation} From the \emph{Toolbox}, select \emph{BlockInstance} and click on the instance element  available on the created diagram to create the a block instance. Select the Block Instance  created and on the \emph{INTO-CPS} property view , change its \emph{Name} to `$controller$', and its \emph{Type} to `$Controller$'

\begin{center}
\begin{annotation}[width=0.7\linewidth]{figures/modelio_controller_instance}
    \usquare{1.5cm}{0.1cm}{0.25}{Create a Block Instance}{0.05,0.76}
    %\helpergrid
    \end{annotation}
\end{center}

\item Select `$controller$' block instance in the model outline and right click. Select `Modeller Module \menusep Update instance or part from an existing classifier'. Click \emph{Update} then \emph{OK}.

\begin{center}
\begin{annotation}[width=0.7\linewidth]{figures/modelio_update_instance}
    %\usquare{2cm}{0.1cm}{0.5}{Update Block Instance}{0.71,0.63}
    %\helpergrid
    \end{annotation}
\end{center}

\item Check the created ports inside the model explorer.

\begin{center}
\begin{annotation}[width=0.7\linewidth]{figures/update-instance-result.png}
    %\usquare{2cm}{0.1cm}{0.5}{Update Block Instance}{0.71,0.63}
    %\helpergrid
    \end{annotation}
\end{center}

\item \label{end-instanciation}Drag and Drop the ports from the model explore inside the diagram to unmask them.

\begin{center}
\begin{annotation}[width=0.7\linewidth]{figures/dd-port-result.png}
    %\usquare{2cm}{0.1cm}{0.5}{Update Block Instance}{0.71,0.63}
    %\helpergrid
    \end{annotation}
\end{center}

\item Repeat from \ref{start-instanciation}  to \ref{end-instanciation} to create block instance of each CComponent.

\begin{center}
\begin{annotation}[width=0.7\linewidth]{figures/instanciation-result.png}
    %\usquare{2cm}{0.1cm}{0.5}{Update Block Instance}{0.71,0.63}
    %\helpergrid
    \end{annotation}
\end{center}

\item From the \emph{Toolbox}, select \emph{Connector} and draw link from Output port to Input to design all connections.

\begin{center}
\begin{annotation}[width=0.7\linewidth]{figures/connection-result.png}
    %\usquare{2cm}{0.1cm}{0.5}{Update Block Instance}{0.71,0.63}
    %\helpergrid
    \end{annotation}
\end{center}

\end{instructions}

\section{Generating Co-Simulation Configuraton}

This section will outline the steps necessary to generate a Co-simulation configuration in Modelio and include the file to the INTO-CPS app.

\begin{instructions}

\item Find any `$CComponent$' block in the model outline and right click. Select `INTO-CPS \menusep Generate Model Description'. Click \emph{Export} then \emph{OK}.

\begin{center}
\begin{annotation}[width=0.7\linewidth]{figures/modelio_gen_cosim}
    %\usquare{2cm}{0.1cm}{0.5}{Export DSE Configuration}{0.71,0.63}
    %\helpergrid
    \end{annotation}
\end{center}


In the INTO-CPS app, this configuration file can be found in `SysML \menusep LineFollowerRobot \menusep config’. Right click the configuration file and select `Create Multi model’.


\end{instructions}

\section{Exporting Model Description}

This section will outline the steps necessary to export a CComponent as a FMI Model Description.

\begin{instructions}

\item Select any CComponent block like the `$Controller$' block in the model outline and right click. Select `INTO-CPS \menusep Generate Model Description'. Click \emph{Export} then \emph{OK}.

\begin{center}
\begin{annotation}[width=0.7\linewidth]{figures/modelio_export_modelDescription.png}
    %\usquare{2cm}{0.1cm}{0.5}{Export Model Description}{0.71,0.63}
    %\helpergrid
    \end{annotation}
\end{center}

\item Choose a file name destination.

\begin{center}
\begin{annotation}[width=0.7\linewidth]{figures/modelio_export_modelDescription_choose_filename.png}
    %\usquare{2cm}{0.1cm}{0.5}{Export Model Description}{0.71,0.63}
    %\helpergrid
    \end{annotation}
\end{center}

\end{instructions}


\section{Importing Model Description}

This section will outline the steps necessary to import an existing Model Description in Modelio.

\begin{instructions}

\item Right click in the Modelio model browser on any \textit{Package} element, then select `INTO-CPS \menusep Import Model Description'.

\begin{center}
\begin{annotation}[width=0.7\linewidth]{figures/sysml-reverse.png}
   % \usquare{2cm}{0.1cm}{0.5}{Import Model DEscription}{0.71,0.63}
    %\helpergrid
    \end{annotation}
\end{center}

\item  Select the desired target  file in your computer and click on \textit{Import}.

\begin{center}
\begin{annotation}[width=0.7\linewidth]{figures/sysml-reverse-selection.png}
   % \usquare{2cm}{0.1cm}{0.5}{Import Model DEscription}{0.71,0.63}
    %\helpergrid
    \end{annotation}
\end{center}

This import command creates an Architecture Structure Diagram describing the interface of an INTO-CPS \emph{CComponent} corresponding to the \texttt{model\allowbreak{}Des\-crip\-tion\allowbreak{}.xml} file imported.

\begin{center}
\begin{annotation}[width=0.7\linewidth]{figures/sysml-reverse-result.png}
   % \usquare{2cm}{0.1cm}{0.5}{Import Model DEscription}{0.71,0.63}
    %\helpergrid
    \end{annotation}
\end{center}
\end{instructions}

\end{document}
