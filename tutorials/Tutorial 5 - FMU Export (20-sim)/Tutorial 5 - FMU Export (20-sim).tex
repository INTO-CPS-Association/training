\documentclass[11pt,a4paper]{../tutorial}
\usepackage{fancyvrb}
\usepackage[hidelinks]{hyperref}

\title{Tutorial 5 --- FMU Export (20-sim)}
\date{March 2018}
\author{Hugo D. Macedo}

\def\intocpsVer{3.4.9}

\newcommand{\tsimurl}{http://www.20sim.com/downloads/files/20sim.exe}

\begin{document}

\section*{Overview}

This INTO-CPS tutorial will show you how to:

\begin{enumerate}[noitemsep]
\item Generate a new physical model simulator FMU in 20-sim
    \begin{enumerate}[noitemsep]
        \item Import a model description into 20-sim
        \item Complete the skeleton model to produce a working simulator
        \item Export the simulator FMU
    \end{enumerate}
\item Associate the new simulator FMU with a multi-model configuration
\item Execute a co-simulation using the new simulator
\end{enumerate}

In the previous tutorials, you have used a pre-exported \emph{Body\_Block} FMU.
In this tutorial, you will find out how that FMU was produced, and you will have the opportunity to modify its behaviour.

\section*{Requirements}

This tutorial requires the following tools from the INTO-CPS tool chain to be installed:

\begin{itemize}[noitemsep]
\item INTO-CPS Application
\item COE (Co-simulation Orchestration Engine) -- accessible through the INTO-CPS App Download Manager
\item 20-sim - Modeling and simulation tool 
\end{itemize}

\section{Download and install the 20-sim tool (Windows only)}

\begin{instructions} 

\item Launch the \emph{INTO-CPS Application}. Download the 20-sim modeling tool from the  \emph{Window \menusep Show Download Manager}. If after the download click \emph{Yes}. If the window does not appear please find the installer inside the \emph{into-cps-projects\pathsep install\_downloads} folder.

\begin{annotation}[width=0.85\linewidth]{figures/Run20simInstaller}
        \dsquare{1.6cm}{0.1cm}{0.75}{Click \emph{Yes\ldots}}{0.785,0.175}	
\end{annotation}

\newpage

\item Follow the installation wizard and make sure to enable Python support.

\item Launch 20-sim, and click \emph{Activation} when the activation screen appears.

\begin{annotation}[width=0.5\linewidth]{figures/Activation1}
        \rsquare{1.3cm}{0.1cm}{0.1}{Click \emph{Activation\ldots}}{0.87,0.125}	
\end{annotation}

\item The following options will appear, click \emph{Next} to continue. 

\begin{annotation}[width=0.5\linewidth]{figures/Activation2}
        \rsquare{1.3cm}{0.1cm}{0.1}{Click \emph{Next\ldots}}{0.72,0.065}	
\end{annotation}

\item Select \emph{I received a license Key by email} and click \emph{Next} to continue. 

\begin{annotation}[width=0.5\linewidth]{figures/Activation3}
        \rsquare{1.3cm}{0.1cm}{0.1}{Click \emph{Next\ldots}}{0.72,0.065}	
\end{annotation}

\newpage 
\item Please fill in the license key you were given (Available in the blackboard system) and click \emph{Finish}.
	
\begin{annotation}[width=0.5\linewidth]{figures/Activation4}
        \rpoint{0.63}{Paste \emph{here\ldots}}{0.5,0.78}	
\end{annotation}

\item You will be asked go online. Click on \emph{Active Now} to finish activation. 
	
\begin{annotation}[width=0.5\linewidth]{figures/WebActivation}
  \rsquare{1.7cm}{0.1cm}{0.1}{Click}{0.33,0.15}	
\end{annotation}

\end{instructions}

\section{Download and install the 20-sim tool (Linux and macOS only)}

It is possible to run 20-sim on Linux and macOS systems after installing Wine.
\begin{itemize}
	\item For the macOS there is a disk image file (.dmg) ready to install. Please ask
us. 
	\item For Linux 
		\begin{enumerate} 
			\item Install the Wine package for your Linux distribution or follow the instructions for your Linux distribution from \url{https://wiki.winehq.org/Download}
    			\item Use the winetricks tool to install a few Microsoft DLLs to fix the license activation and some drawing issues:
        			\begin{itemize}
					\item winetricks msxml4
					\item winetricks wininet
					\item winetricks gdiplus 
				\end{itemize}
			\item Now download 20-sim from \url{\tsimurl} and install it using: \verb'wine 20sim.exe'
		\end{enumerate}

This should result in a working 20-sim version on Linux. Tested here with various Debian and Ubuntu versions and Wine version 1.5 up to Wine version 3.0.
\end{itemize}

\newpage 

\section{Loading a Model in 20-sim}

\begin{instructions}

\item \label{step:exp3} Launch the \emph{20-sim} tool. You should see the following 20-sim window.

    \begin{annotation}[width=0.85\linewidth]{figures/20-simWelcome}
    \end{annotation}


\item Open the 20-sim physical model of the line follower robot by selecting \emph{File \menusep Open}.

     \begin{annotation}[width=0.8\linewidth, trim={0 12cm 0 0},clip]{figures/20-simNewOpen}
       % \upoint{0.37}{1. Locate and select \emph{Body\_Block.fmu}}{0.33,0.7}
       % \dsquare{1.9cm}{0.12cm}{0.63}{2. Click \emph{Open}}{0.74,0.055}
    \end{annotation}



\item \label{step:exp4} Navigate to the location of \emph{tutorial\_5\pathsep{}Models\pathsep{}20-sim\_Body} and \emph{Open} the \emph{R2G2P\_Body\_Only} model. 

    \begin{annotation}[width=0.4\linewidth]{figures/BodyBlockModel}
       % \upoint{0.37}{1. Locate and select \emph{Body\_Block.fmu}}{0.33,0.7}
       % \dsquare{1.9cm}{0.12cm}{0.63}{2. Click \emph{Open}}{0.74,0.055}
    \end{annotation}

\newpage

\item You should see the block diagram for the model. The diagram is composed of two blocks: The \emph{Environment} and the \emph{Body\_Block}. 
The Body\_Block corresponds to the \emph{Body\_Block} FMU, and the Environment provides default inputs to enable the simulation of the Body\_Block. Once the Body\_Block FMU is exported, the inputs to the Body\_Block will be provided by the co-simulation.
Click on the Body\_Block to visualize its contents.  

    \begin{annotation}[width=0.8\linewidth]{figures/20-simModelTop}
       \rsquare{1.05cm}{0.6cm}{0.2}{Click \emph{Body\_Block}}{0.73,0.57}
    \end{annotation}

\newpage

\item At this step you can explore the model further by clicking on blocks or
	by clicking on the Model view. To continue our tutorial open the
	\emph{wheel\_right} submodel.

    \begin{annotation}[width=0.8\linewidth]{figures/20-simModelBody}
       % \upoint{0.37}{1. Locate and select \emph{Body\_Block.fmu}}{0.33,0.7}
       \dsquare{0.16cm}{0.8cm}{0.5}{Click the wheel\_right}{0.642,0.546}
       \dpoint{0.5}{}{0.1,0.575}
    \end{annotation}

\newpage

\item The \emph{wheel\_right} models the physical behaviour of the line
	following robot wheel. You can explore the model but we will focus on
	the \emph{rotation\_to\_translation} element. This element is a
	transformer which as it implied by its name transforms the rotation of
	the wheel into the translation of the line following robot body
	structure.  

    \begin{annotation}[width=0.8\linewidth]{figures/20-simWheel1}
       % \upoint{0.37}{1. Locate and select \emph{Body\_Block.fmu}}{0.33,0.7}
       % \dsquare{1.9cm}{0.12cm}{0.63}{2. Click \emph{Open}}{0.74,0.055}
       \dsquare{0.8cm}{0.8cm}{0.55}{rotation\_to\_translation}{0.652,0.58}
%       \helpergrid
    \end{annotation}

\item After double clicking the transformer you should find its contents to be empty.

    \begin{annotation}[width=0.7\linewidth]{figures/20-simWheel2}
       % \upoint{0.37}{1. Locate and select \emph{Body\_Block.fmu}}{0.33,0.7}
       % \dsquare{1.9cm}{0.12cm}{0.63}{2. Click \emph{Open}}{0.74,0.055}
    \end{annotation}

\newpage

\item Fill in the \emph{rotation\_to\_translation} with the content shown on the figure, and save afterwards.

    \begin{annotation}[width=0.8\linewidth]{figures/20-simWheel3}
       \upoint{0.1}{1. Fill in }{0.6,0.8}
       \dsquare{0.5cm}{0.12cm}{0.63}{2. Click \emph{Save}}{0.12,0.92}
	    
    \end{annotation}

\end{instructions}

\bigskip
\bigskip
{\large\bfseries Congratulations!}

You edited your first 20-sim physical simulator.

In the following sections, you'll need either a Windows system or an emulated Windows platform, to be
able to export an FMU for the physical model of the Line Following Robot. The
simulator FMU can be provided to you and you can continue in Section
\ref{sec:cosim}. If you cannot follow along in your computer, join one of your colleagues.

\newpage

\section{Installing Microsoft Build Tools (Windows Only)} 
\label{sec:install_msbuild}

20-sim is able to export new FMUs however \emph{this feature depends on Microsoft Build Tools being available}, because 20-sim exports the source code of the FMU and then builds it using Microsoft Build Tools.
If you suspect you already have Microsoft Build Tools installed, try to follow the instructions in Section~\ref{sec:export_fmu}.
If 20-sim fails to build the FMU, then refer to the following step as one possible way to install Microsoft Build Tools on your system.

\begin{instructions}

\item Install (or modify the existing installation of) Visual Studio Community 2019, and include \emph{Desktop Development with C++}.

    \begin{annotation}[width=0.8\linewidth,trim=0 0cm 0 0,clip ]{figures/msbuild_cpp_install}
        \dsquare{4.0cm}{0.2cm}{0.63}{Check}{0.51,0.45}	    
    \end{annotation}


\end{instructions}

\newpage

\section{Exporting an FMU and Adding it to a Multi-model}
\label{sec:export_fmu}

\begin{instructions}

\item \label{step:exp1} Reload the 20-sim model again by repeating \ref{step:exp3} to \ref{step:exp4} and click \emph{Start Simulator} to open the Simulator Window.

    \begin{annotation}[width=0.8\linewidth]{figures/20-simModelTop}
	   \usquare{0.6cm}{0.1mm}{0.37}{Click on \emph{Start Simulator}}{0.5,0.91}
    \end{annotation}

\newpage

\item After the \emph{Simulator Window} appears, navigate to \emph{Tools\menusep Real Time Toolbox\menusep C-Code Generation} and click \emph{C-Code Generation}.

   \begin{annotation}[width=0.8\linewidth]{figures/20-simCCodeGen}
        \dsquare{3.1cm}{0.06cm}{0.63}{Click \emph{C-Code Generation}}{0.743,0.71}	    	   
    \end{annotation}

\item Select \emph{FMU 2.0 export for 20-sim submodel} from the \emph{Target List} and click on the \emph{Submodel} icon.

   \begin{annotation}[width=0.5\linewidth]{figures/20-simCCodeGenFMI20}
        \upoint{0.01}{1. Click on \emph{FMU 2.0 export\ldots}}{0.2,0.57}	    	   
        \dsquare{1.2cm}{1cm}{0.8}{2. Click on the \emph{Submodel} icon}{0.59,0.26}	    	   
	   
   \end{annotation}

\newpage

\item A model chooser appears. Select the \emph{Body\_block} in the Model Hierarchy. 

   \begin{annotation}[width=0.5\linewidth]{figures/20-simCCodeGenModel}
	\upoint{0.63}{1. Select \emph{Body\_ Block}}{0.24,0.59}	    	   
        \dsquare{2cm}{0.6cm}{0.63}{2. Click \emph{OK}}{0.9,0.14}	
   \end{annotation}

\item You may change the location for the FMU output by changing the \emph{Output Directory} or leave it to its default location and click \emph{OK} to generate the FMU. 

   \begin{annotation}[width=0.5\linewidth]{figures/20-simCCodeGenExport}
        \dsquare{1.3cm}{0.4cm}{0.63}{Click \emph{OK}}{0.89,0.185}	
   \end{annotation}

\newpage

\item The following shell should pop up and the script to be run may take a while to complete.

   \begin{annotation}[width=0.8\linewidth]{figures/20-simCCodeGenScript}
   \end{annotation}

\item When the script is finished and the FMU generation succeeded you should see the following message.

   \begin{annotation}[width=0.8\linewidth]{figures/20-simCCodeGenScriptOK}
   \end{annotation}

	Common issues at this step:
	\begin{itemize}
		\item 20-sim is unable to collect resources -- This might be caused by python not having been installed when you ran the 20-sim setup. Try reinstalling 20-sim with that support enabled.
		\item 20-sim cannot determine the location of the VS Common Tools folder -- There are multiple possible causes for this error:
		\begin{enumerate}
			\item You don't have Microsoft Build Tools installed, and/or
			\item The script \emph{vsvars32.bat} expects to find a registry entry that contains the path to the Common tools folder (e.g., \emph{C:\textbackslash Program Files (x86)\textbackslash Microsoft Visual Studio 14.0\textbackslash Common7 \textbackslash Tools}). This issue is described in \footnote{\url{https://stackoverflow.com/questions/3461275/vs2010-command-prompt-gives-error-cannot-determine-the-location-of-the-vs-commo}}. The easiest solution is to uninstall Microsoft Build Tools 2015 follow the instructions in Section~\ref{sec:install_msbuild}.
		\end{enumerate}
	\end{itemize}

\clearpage

\item Navigate to the chosen folder and copy the \emph{Body\_Block.fmu}.

   \begin{annotation}[width=0.8\linewidth]{figures/GetGenFMU}
   \end{annotation}

\item \label{step:exp2}  Paste the \emph{Body\_Block.fmu} into the FMUs folder inside tutorial 5..

   \begin{annotation}[width=0.8\linewidth]{figures/PasteGenFMU}
   \end{annotation}

Congratulations! You are now ready to use the generated FMU in a co-simulation. 
\end{instructions}


\newpage

\section{Co-simulating with the New Simulator}
\label{sec:cosim}

\begin{instructions}
\item Launch the \emph{INTO-CPS Application} and select \emph{File \menusep Open Project}. Set the \emph{Project root path} to the location of \emph{Tutorials\pathsep{}tutorials\_5} and click \emph{Open}. You can browse using the \emph{Folder} button.

    \begin{annotation}[width=0.35\linewidth,trim=0 0 0 0,clip]{figures/projectBrowser4}
    \upoint{0.45}{Path to \emph{Tutorials\pathsep{}tutorials\_5}}{0.33,0.7}
    %\helpergrid
    \end{annotation}

    You should see the newly export \emph{Body\_Block} FMU in the list.

    \begin{annotation}[height=0.5\linewidth,trim=0 125 250 0,clip]{figures/INTOnewFMU}
        \lpoint{0.69}{Newly created FMU}{0.06,0.69}
    \end{annotation}

\item In the SysML entry of the project browser, expand the \emph{LineFollowRobot} folder, then \emph{config} folders. Right-click on \emph{3DRobot} and select \emph{Create Multi-Model}.

    \begin{annotation}[width=0.35\linewidth,trim=0 40 375 300,clip]{figures/into_create_mm}
        \rpoint{0.7}{Expand to locate \emph{3DRobot}}{0.53,0.63}
        \rsquare{2.7cm}{0.5cm}{0.5}{Create Multi-model}{0.68,0.5}
    \end{annotation}

\newpage
\item We now need to associate FMUs to the multi-model as we did in \emph{Tutorial 2}. Scroll down to find the \emph{Configuration} panel and expand it by clicking the arrow.

    \begin{annotation}[width=0.85\linewidth,trim=0 0 0 250,clip]{figures/INTOExpandConfig}
        \dcircle{0.15cm}{0.8}{Expand \emph{Configuration}}{0.938,0.148}
    \end{annotation}

\item Scroll down and click \emph{Edit}.

    \begin{annotation}[width=0.85\linewidth,trim=0 0 0 260,clip]{figures/INTOEditConfig}
        \dsquare{1.1cm}{0.12cm}{0.8}{\emph{Edit} configuration}{0.332,0.455}
    \end{annotation}

\item As in \emph{Tutorial 2}, in the FMUs section press \emph{File} next to the Body\_ Block element, \emph{b}. A file browser window will open and show five FMUs (if the file browser does not show the FMUs, navigate to \emph{tutorials\_5\pathsep{}FMUs}). Select \emph{Body\_Block.fmu} and click \emph{Open}.

    \begin{annotation}[width=0.8\linewidth]{figures/FMUFindFile}
        \upoint{0.37}{1. Locate and select \emph{Body\_Block.fmu}}{0.33,0.7}
        \dsquare{1.9cm}{0.12cm}{0.63}{2. Click \emph{Open}}{0.74,0.055}
    \end{annotation}

\item Repeat this for the remaining elements:
    \begin{itemize}
        \item \emph{c} : \emph{LFRController.fmu}
        \item \emph{3D} : \emph{3DanimationFMU.fmu}
        \item \emph{sensor1} : \emph{Sensor\_Block\_01.fmu}
    \item \emph{sensor2} : \emph{Sensor\_Block\_02.fmu}
    \end{itemize}

    The complete set of FMUs will look like this:

    \begin{annotation}[width=0.85\linewidth,trim=0 0 0 0,clip]{figures/INTOFMUsDone}
    \usquare{9cm}{12.7cm}{0.58}{FMUs added}{0.616,0.47}
    \end{annotation}

\newpage
\item Scroll down to the \emph{Initial values of parameters section}, and click \emph{\{sensor1\}.sensor1}. In the \emph{Parameters} section, enter the following values:

    \begin{itemize}
        \item \emph{lf\_position\_y} = 0.065
        \item \emph{lf\_position\_x} = 0.01
    \end{itemize}

    \begin{annotation}[width=0.8\linewidth]{figures/INTOParams}
        \dpoint{0.1}{\emph{lf\_position\_y}}{0.28,0.28}
        \dpoint{0.5}{\emph{lf\_position\_x}}{0.42,0.15}
    \end{annotation}

\item Repeat the previous step for the second sensor, \emph{\{sensor2\}.sensor2}, with the following values:

    \begin{itemize}
        \item \emph{lf\_position\_x} = -0.01
        \item \emph{lf\_position\_y} = 0.065
    \end{itemize}

\item \emph{Save} the \emph{Configuration}.

    \begin{annotation}[width=0.85\linewidth,trim=0 0 0 250,clip]{figures/INTOSaveConfig}
        \dsquare{1.25cm}{0.12cm}{0.5}{\emph{Save} configuration}{0.338,0.18}
    \end{annotation}

\item Right-click on the new multi-model configuration and select \emph{Create Co-simulation Configuration}.

    \begin{annotation}[width=0.85\linewidth,trim=0 120 0 130,clip]{figures/INTOCreateCoSim}
        \dsquare{3.85cm}{0.12cm}{0.5}{\emph{Create Co-Simulation Configuration}}{0.293,0.52}
    \end{annotation}

\item Set the \emph{Step size} to 0.01. Don't forget to press \emph{Edit} then \emph{Save}.

    \begin{annotation}[width=0.85\linewidth,trim=0 0 0 0,clip]{figures/INTOSetSimTime}
        \usquare{1.25cm}{0.12cm}{0.3}{\emph{Edit} then \emph{Save}}{0.338,0.722}
        \ucircle{0.15cm}{0.8}{Expand \emph{Configuration}}{0.94,0.795}
        \dpoint{0.7}{Set \emph{Step size}}{0.64,0.08}
    \end{annotation}

\item Check \emph{lf\_1\_sensor\_reading} from \emph{\{sensor1\}.sensor1} and \emph{\{sensor2\}.sensor2} to see the sensor values appear in the \emph{Live Plotting}.

    \begin{annotation}[width=0.85\linewidth,trim=0 230 0 250,clip]{figures/INTOPlotSensor}
    \end{annotation}

\item Launch the COE if necessary (see \emph{Tutorial 1 --- First Co-simulation} for a reminder if needed).

    \begin{annotation}[width=0.85\linewidth,trim=0 270 0 120,clip]{figures/INTOLaunchCOE}
        \usquare{1.25cm}{0.12cm}{0.8}{\emph{Launch} COE}{0.865,0.42}
    \end{annotation}

\item When the COE is running, click the \emph{Simulate} button. If you are on Windows, the \emph{Animation Frame} should appear. You can click the \emph{3D} button to see the 3D visualisation of the robot.

    \begin{annotation}[width=0.55\linewidth,trim=0 0 0 0,clip]{figures/into_coe_3d_toggle}
        \rcircle{0.32cm}{0.8}{3D Button}{0.52,0.88}
    \end{annotation}

\item If everything went well, the robot should follow the line as in \emph{Tutorial 2 --- Adding FMUs}. Although with a slight \emph{difference}\ldots

    \begin{annotation}[width=0.55\linewidth,trim=0 300 0 0,clip]{figures/into_coe_3d_results}
    \end{annotation}

\end{instructions}

\section{Additional Exercises}

\begin{enumerate}
	\item Can you spot the \emph{difference} between the co-simulation you run in this tutorial and the co-simulation you run in the previous tutorial? You can go back to \emph{20-sim} and explore\footnote{Ask for a hint in case you feel lost\ldots} the physical model. Try to make some changes to obtain the correct behaviour. Just repeat \ref{step:exp1} to \ref{step:exp2} to regenerate and copy the FMU, then press \emph{Simulate}.
\end{enumerate}



\end{document}
