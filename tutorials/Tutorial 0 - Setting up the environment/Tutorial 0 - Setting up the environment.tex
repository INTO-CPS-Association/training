\documentclass[11pt,a4paper]{../tutorial}
\usepackage[hidelinks]{hyperref}

\title{Tutorial 0 --- Setting up the environment}
\date{March 2018}
\author{Hugo D. Macedo}

\def\intocpsVer{4.0.0}
\newcommand{\JavaURL}{https://www.oracle.com/java/technologies/javase-jdk11-downloads.html}
\newcommand{\JavaWinVer}{jdk-11.0.9\_windows-x64\_bin.exe}

\begin{document}

\section*{Overview}

This INTO-CPS tutorial will show you how to:

\begin{enumerate}[noitemsep]
\item Download and install the INTO-CPS tool
\item Install the COE (Co-simulation Orchestration Engine)
\item Uninstall the tool (Optional) 
\end{enumerate}

\section*{Requirements}

Although this tutorial is self-contained and depends only on a standard
installation of a Linux, Windows, or macOS system, you need to ensure some
dependencies are installed in your system in order to be able to fully use the
INTO-CPS Application. Namely: 

\begin{itemize}[noitemsep]
	\item Java SE Runtime Environment 8. It is recommended to install from the following files: 
	\begin{itemize}
	\item win64 -- \JavaWinVer, which can be downloaded from \url{\JavaURL} 
	\item linux64 -- jdk-11.0.9\_linux-x64\_bin.deb
	\end{itemize}
\item Git
\end{itemize}


\section{Download and install the INTO-CPS tool}


\begin{instructions} 

\item The INTO-CPS Application can be downloaded from
	\url{https://into-cps-association.github.io/download/}. Unzip the
	contents in a directory and inside it you will find the into-cps-app-\intocpsVer\
	executable you need to launch in the next step. 
	
	\begin{annotation}[width=0.85\linewidth]{figures/downloadPage}
        	\usquare{1.6cm}{0.25cm}{0.15}{Select version}{0.48,0.665}
	%	\helpergrid
	\end{annotation}

\end{instructions}

\newpage 

\section{Install the COE (Co-simulation Orchestration Engine)}

\begin{instructions}
\item Launch the \emph{INTO-CPS Application}. On first loading, it will look like the screenshot below. If you have opened a project previously, that project will be opened automatically.

\begin{annotation}[width=0.85\linewidth]{figures/blankApp}
\end{annotation}

\item To install a tool, select \emph{Window \menusep Show Download Manager}.

\begin{annotation}[width=0.85\linewidth,trim=0 260 0 0,clip]{figures/showDownloadManager}
    \usquare{3.8cm}{0.3cm}{0.6}{Show Download Manager}{0.28,0.5}
\end{annotation}

\newpage
\item A new window with release options appear. Press the latest release (the one on the top).

\begin{annotation}[width=0.85\linewidth,trim=0 160 0 0,clip]{figures/releasesWindow}
    \usquare{10.7cm}{1.6cm}{0.7}{Latest release}{0.48,0.4}
\end{annotation}


\item Once the release is opened, you will see the various tools that can be installed using the INTO-CPS Application. To install the COE please find the ``Co-simulation Orchestration Engine - Co-simulation Orchestration Engine'' entry and press the down arrow and confirm. 

\begin{annotation}[width=0.85\linewidth]{figures/coeInRelease}
    \usquare{0.54cm}{0.3cm}{0.6}{Download COE}{0.565,0.065}
\end{annotation}

\newpage
\item  When the COE is downloaded the progress bar stalls on 100\% as shown in the figure. You should now click on the Close button.    

\begin{annotation}[width=0.85\linewidth]{figures/fullDownload}
    \usquare{0.9cm}{0.3cm}{0.1}{Close window}{0.033,0.03}
\end{annotation}


    \bigskip
    \bigskip
    {\large\bfseries Congratulations!}

    You have installed the INTO-CPS Application and the COE. Your system is ready for the next tutorial.

\end{instructions}

\newpage
\section{Uninstall tool (Optional) [In case of emergency]}

To completely remove the tool from your system you must remove the folder
created when unzipping the tool during the installation process and the
following folders: 

\begin{instructions}
\item Remove the \verb'into-cps-projects' folder which contain all the projects and downloaded tools. The following are the shell commands you should run depending on your operating system:  
	\begin{itemize}
		\item Linux: \verb'rm -rf ~/into-cps-projects/'
		\item Windows: \verb'rm -Recurse -Force $HOME\Documents\into-cps-projects'	
		\item macOS: \verb'rm -rf ~/into-cps-projects/'
	\end{itemize}

\item Remove the configurations folder. The following are the shell commands you should run depending on your operating system:  

	\begin{itemize}
		\item Linux: \verb'rm -rf ~/.config/INTO-CPS\ APP/'
		\item Windows: \verb'rm -Recurse $HOME\AppData\Roaming\INTO-CPS APP'
		\item macOS: \verb' rm -rf ~/Library/Application Support/INTO-CPS APP/'
	\end{itemize}
\end{instructions}

\end{document}
