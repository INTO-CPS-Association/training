\documentclass[11pt,a4paper]{../tutorial}
\usepackage[hidelinks]{hyperref}

\title{Tutorial 10 --- Deploying the LFRController}
\date{March 2018}
\author{Hugo D. Macedo}

\def\intocpsVer{3.4.6}
\begin{document}

\section*{Overview}

This INTO-CPS tutorial will show you how to:

\begin{enumerate}[noitemsep]

\item Export the \emph{LFRController} VDM project as an FMU
\item Prepare an ARDUINO Sketch with the FMU code 
\item Compile and upload the Sketch into the robot 
\end{enumerate}

The instructions in this tutorial were adapted from \url{https://github.com/INTO-CPS-Association/example-line\_follower_robot/tree/arduinoMegaDeploy/deployment}. 

\section*{Requirements}

This tutorial requires the following tools from the INTO-CPS tool chain to be installed:

\begin{itemize}[noitemsep]
\item Overture FMU Import/Exporter CLI
\end{itemize}

In addition, the following software must be installed without recurring to the Download Manager: 

\begin{itemize}[noitemsep]
\item ARDUINO IDE (1.8.5) 
\item avr-gcc (GCC) $\geq$ 6.3.0
\item avr-g++ (GCC) $\geq$ 6.3.0   
\end{itemize}

In terms of hardware one needs a printed version of the line map and the robot:

\begin{annotation}[width=0.85\linewidth,trim={20cm 20cm 20cm 20cm},clip]{figures/robotUnplugged}
\end{annotation}

\newpage
\section{Download and install the dependencies}


\begin{instructions} 

\item Install the Overture FMU Import/Exporter CLI using the INTO-CPS
	application \emph{Download Manager}. Please make sure you download the
	\textbf{CLI} version. After the download you will find a new .jar file it in your \texttt{into-cps-projects/install\_downloads}. 
	
	\begin{annotation}[width=0.85\linewidth]{figures/downloadedcli}
        	\upoint{0.15}{New jar}{0.33,0.35}
	%	\helpergrid
	\end{annotation}

	\newpage
\item Download the ARDUINO IDE version 1.8.5 from:\\
	\url{https://www.arduino.cc/en/Main/Software}.  Choose the zip file
	version and extract the folder into your
	\texttt{into-cps-projects/install} folder. (Choose the respective files in the case of Linux or macOS installations)

	\begin{annotation}[width=0.85\linewidth, trim= {0 6cm 0 0},clip]{figures/ArduinoIDE}
        \upoint{0.15}{Download the Zip File\ldots}{0.7,0.51}
		%\helpergrid
	\end{annotation}


\item The current C/C++ compiler version shipped in the ARDUINO IDE is prior to the required 6.3. So one needs to install an updated version. You can find one from:\\
	\url{http://blog.zakkemble.co.uk/avr-gcc-builds/}. 
	
	Download and unzip it into your  \texttt{into-cps-projects/install} folder. (Choose the respective files in the case of Linux or macOS installations)


	\begin{annotation}[width=0.85\linewidth, trim= {0 6cm 0 0},clip]{figures/GCCDownload}
        \upoint{0.15}{Download the Zip File\ldots}{0.15,0.73}
		%\helpergrid
	\end{annotation}

\newpage
\item At this point your \texttt{into-cps-projects/install} folder should contain the two folders corresponding to the ARDUINO IDE and the updated version of the compiler. And you are ready for the next step. 

	\begin{annotation}[width=0.85\linewidth]{figures/ArduinoInInstall}
        	\upoint{0.15}{ARDUINO IDE}{0.23,0.5}
		\upoint{0.8}{Compiler}{0.4,0.54}
		%\helpergrid
	\end{annotation}
\newpage
\end{instructions}



\section{Export the LFRController VDM project as an FMU}

\begin{instructions} 

\item Open a Command Prompt. Change directory into  \emph{Deploy\pathsep linux} or (\emph{Deploy\pathsep win64)}   inside the \emph{tutorial 10}. It is important you run the command below in this folder or copy the output file produced (\emph{lf.fmu})  into it. If the command succeeds you should see a new file in the folder as depicted.
	\begin{verbatim}
	java -jar path_to_install_downloads/overture-fmu-cli.jar 
	     -export source 
	     -name lf 
	     -output . 
	     -root ../../Models/LFRController
	\end{verbatim}

	\begin{annotation}[width=0.85\linewidth]{figures/LfFMU}
        	\upoint{0.15}{New file}{0.23,0.62}
		%\helpergrid
	\end{annotation}

\end{instructions}

\newpage

\section{Compile the FMU and upload to Arduino (Linux/macOS Only)}

\begin{instructions} 

\item Connect robot to USB and find port (for example you may find /dev/ttyACM0 after running \verb'ls /dev/' ). 

	
	\begin{annotation}[width=0.85\linewidth,trim={10cm 0 10cm 10cm},clip]{figures/robotPlugged}
	\end{annotation}

\item In a terminal change to the \emph{tutorial\_10/Deploy/linux} folder. In it you should find the files: \emph{main.cpp}, \emph{modeldescription.h} and the \emph{deploy.sh} script. 

	\begin{annotation}[width=0.85\linewidth,trim={0 30cm 0 0},clip]{figures/deployContents}
		%\helpergrid
	\end{annotation}

	
\item Set the following variables used in the \emph{deploy.sh} script adapting the path to your own choices:
	\begin{itemize}
		\item port - to the result of the previous step
		\item gcc\_path - to into\_cps\_project/install/avr-gcc-7.3.0-x64-linux/bin 
		\item avr - to into\_cps\_project/install/arduino-1.8.5/hardware/arduino/avr/ 
		\item avrdudeconfig - to into\_cps\_project/install/arduino-1.8.5/hardware/tools/avr/etc/avrdude.conf
	\end{itemize}
Run the \emph{deploy.sh} script with  lf.fmu as a parameter.

		\begin{annotation}[width=0.85\linewidth,trim={10cm 35cm 15cm 0},clip]{figures/RunDeploy}
	\end{annotation}

\newpage

\item You should see the following message in case everything works as expected. 

	\begin{annotation}[width=0.85\linewidth]{figures/avrdudeOK}
	\end{annotation}


\item Unplug the USB cable, place the robot in the line, turn the motor switch on!
	
	\begin{annotation}[width=0.85\linewidth,trim={10cm 0 10cm 10cm},clip]{figures/robotOver}
		        	\upoint{0.15}{Motor On/Off}{0.5,0.5}
	%	\helpergrid
	\end{annotation}
\end{instructions}


\section{Compile the FMU and upload to Arduino (Windows Only)}


\begin{instructions} 
\item Connect robot to USB and find port (for example you may find COM3 by typing \verb'mode') in a Command Prompt or check if the ARDUINO IDE detects the port automatically 

\item Set the following variables used in the \emph{deploy.sh} script adapting the path to your own choices:
	\begin{itemize}
		\item gcc\_path - to into\_cps\_project/install/avr-gcc-7.3.0-x64-linux/bin 
		\item avr - to into\_cps\_project/install/arduino-1.8.5/hardware/arduino/avr/ 
		\item avrdudeconfig - to into\_cps\_project/install/arduino-1.8.5/hardware/tools/avr/etc/avrdude.conf
	\end{itemize}

\item Fill in the missing details in \emph{deploy.sh} script and run with the port as a parameter. 

\end{instructions}

\section{Additional Exercises}

\begin{instructions} 

\item Browse the different components descriptions you can find inside the \emph{platform} folder in the repository:\\
\url{https://github.com/INTO-CPS-Association/example-line\_follower\_robot/tree/arduinoMegaDeploy/deployment}

\item Workout the details of the hardware platform.

\item Find the relation between the VDM LFRController and the contents of the \emph{main.cpp} file.

\item Currently the robot has an extra mounted and unused sensor\ldots How could you use it to improve the robot behaviour?

\item Find the STL file for the sensor mount:\url{https://github.com/INTO-CPS-Association/example-line_follower_robot/blob/arduinoMegaDeploy/deployment/platform/LF_sensor_holder.stl} How could you use it in future iterations/improvements of the robot? 

\end{instructions}

    \bigskip
    \bigskip
    {\large\bfseries Congratulations!}

    You have reached the end of this tutorial session! 


\end{document}
